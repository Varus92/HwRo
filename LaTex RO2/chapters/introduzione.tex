
% \section{Introduzione}
% Introduzione al corso

\section{Syllabus}

\begin{itemize}
    \item Travelling Salesman Problem (TSP)
    




\end{itemize}

\section{Riferimenti}

Il libro di riferimento è 

\section{Corso}
Il corso di Ricerca Operativa 2 andrà ad analizzare gli argomenti trattati nel corso di Ricerca Operativa 1, andando a svilupparli e trattare delle istanze Definite dei vari problemi.
I problemi che tratteremo è associabile un grafo G=(V,E), in cui V è l'insieme degli n nodi e A è l'insieme degli archi. Si indica con $e\in E$  il costo dell'arco per andare dal nodo i al nodo j, i problemi saranno non simmetrici e tratteremo problemi con grafi orientati.
Quando viene affrontato un problema di questo genere, bisogna specificare la funzione obiettivo che rappresenta la minimizzazione del costo del cammino.
\begin{center}
$ min \sum_{e \in E} Ce Xe $ 
\end{center}

 I vincoli di grado che utilizzeremo saranno: 
\begin{center}
$ \sum {e \in \delta(j)} Xe =2 , \forall v \in V$
\end{center}



